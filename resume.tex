\documentclass[10pt]{article}
\usepackage[letterpaper, margin=0.4in]{geometry}
\usepackage{titlesec}
\usepackage{changepage}

\pagenumbering{gobble}
\hyphenpenalty=10000
\setcounter{secnumdepth}{0} % Suppress section numbers
\titlespacing\section{0pt}{8pt}{4pt}

\begin{document}

\noindent\begin{minipage}[t]{0.5\textwidth}
\huge \textbf{Jonathan G. Tao}
\end{minipage}
\begin{minipage}[b]{0.5\textwidth}
\small
\flushright
jon.g.tao@gmail.com\hspace{16pt}
http://jgtao.me\hspace{16pt}
(415) 812-0180
\end{minipage}
\noindent\rule{\linewidth}{0.05mm}

\section{Education}
\begin{adjustwidth}{16pt}{0pt}
\noindent\textbf{Bachelor of Science in Electrical Engineering}
 - Expected March 2016\\
\noindent\textbf{Bachelor of Science in Computer Engineering}
 - Expected March 2016\\
University of California, Davis \hspace{4pt} with \hspace{4pt}  GPA: 3.84/4.00
(Engineering Dean's Honor List, 8 quarters)

\vspace{8pt}
\noindent\textbf{Relevant Course Work}
\begin{itemize}
	\setlength\itemsep{0pt}
	\item Digital System Design and FPGAs, Analog Circuit Analysis and
	Design, Device Physics, Electricity and Magnetism, Probabilistic Analysis of
	Electrical and Computer Systems, Signals and Systems, Embedded Systems.
	\item Software Development and C/C++ Programming, Discrete Mathematics, Data
	Structures, Algorithm Design and Analysis, Computer Architecture.
	\item Technical Writing in Engineering.
\end{itemize}
\end{adjustwidth}

\section{Skills}
\begin{adjustwidth}{16pt}{0pt}
\begin{itemize}
	\setlength\itemsep{0pt}
	\item Use of oscilloscopes, signal generators, spectrum analyzers,
	multimeters, and logic analyzers to verify circuits.
	\item Printed circuit EDA (CadSoft Eagle, Kicad) with final assembly and
	testing of circuit boards.
	\item Soldering of both through-hole and SMT components with hand tools and
	oven.
	\item Hardware description and testing with SPICE (PSPICE, LTSPICE, HPSPICE)
	and Verilog HDL (Altera Quartus).
	\item Programming in C/C++, MIPS/ARM assembly, MATLAB/Octave, and Bourne
	shell.
	\item Embedded system software development for Cypress PSoC, Texas Instruments
	MSP, Atmel AVR, and ARM Cortex platforms. Familiarity with embedded
	communication protocols and peripheral devices using device registers and API.
	\item Development workflow with Git, Make, and platform specific IDEs (e.g.
	Keil). Familiar with command line environments.
	\item Typesetting in \LaTeX \hspace{3pt} (this document was written with
	\LaTeX) and HTML.
	\item Professional working proficiency in Cantonese Chinese.
\end{itemize}
\end{adjustwidth}

\section{Experience and Projects}
\begin{adjustwidth}{16pt}{0pt}

\noindent\textbf{New Product Introduction Electrical Eng.} - Internship at
Keysight Technologies (Formerly Agilent Tech.)
\hfill 2015
\begin{itemize}
	\setlength\itemsep{0pt}
	\item Analyzed and verified power sequencing, reliability, and design of a PXI
	vector signal generator with SPICE simulations to provide real-time feedback
	for design engineers.
	\item Used BASH and Make to optimize the simulation workflow. Created new
	metrics to quantify simulation credibility and programs to evaluate
	simulations with those metrics. Wrote documentation for existing and new
	software.
\end{itemize}


\vspace{8pt}
\noindent\textbf{Electric Vehicle Management Electronics and Telemetry} - Senior Design Project
\hfill 2014 - 2015
\begin{itemize}
	\setlength\itemsep{0pt}
	\item Created a electric vehicle battery management system, which protected
	cells by monitoring voltage and temperature.
	\item Built a CAN bus sensor network logger with a Cypress PSoC and wireless
	telemetry to a desktop application.
\end{itemize}



\vspace{8pt}
\noindent\textbf{Formula SAE Student Electric} - Race Car Design Competition
\hfill 2013 - Present
\begin{itemize}
	\setlength\itemsep{0pt}
	\item The team took 3rd place at the SAE Electric International competition of
	2014 in Lincoln, Nebraska.
	\item Programmed a supervisory control unit for managing power-up, shut-down,
	driving modes, and emergencies.
	\item Created a driver dashboard interface with an Atmel AVR for monitoring
	vehicle information from a CAN bus.
	\item Developed a KS-108 LCD driver for a dashboard interface with font and
	geometry rendering.
	\item Wrote a PNG image to C constant conversion tool to program fonts and
	images into the dashboard interface. 
\end{itemize}


\vspace{8pt}
\noindent\textbf{Proximity Boxes} - Maker Faire Project 
\hfill 2015
\begin{itemize}
	\setlength\itemsep{0pt}
	\item Created a modular interactive surface that senses a user's proximity
	using modulated infrared reflections. The boxes vary the color of the
	several hundred LEDs on their surfaces to indicate position and proximity of a
	detected object.
	\item Used a central MSP430 controller with I\textsuperscript{2}C expanders to
	manage hundreds of I/O and pulse measurements in real time.
\end{itemize}


\vspace{8pt}
\noindent\textbf{Ludum Dare 28} - 48 Hour Game Development Competition
\hfill 2013
\begin{itemize}
	\setlength\itemsep{0pt}
	\item Within 48 hours, wrote a sandbox game from scratch about killer bunnies
	and holy hand grenades using C++ with the Allegro library, and then ported the
	game from Linux to Windows for cross-platform distribution.
\end{itemize}


\end{adjustwidth}
\end{document}
